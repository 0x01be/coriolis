\subsubsection{Name}

LogicIn, LogicOut ... -- Creation of nets

\subsubsection{Synopsys}

\begin{verbatim}
netA = LogicIn ( "a", 4 )
\end{verbatim}

\subsubsection{Description}

How to create and use nets.

\subsubsection{Nets}

Differents kind of nets are listed below :
\begin{itemize}
    \item \verb-LogicIn- : Creation of an input port
    \item \verb-LogicOut- : Creation of an output port
    \item \verb-LogicInOut- : Creation of an inout port
    \item \verb-LogicUnknown- : Creation of an input/output port which direction is not defined
    \item \verb-TriState- : Creation of a tristate port
    \item \verb-CkIn- : Creation of a clock port
    \item \verb-VddIn- : Creation of the vdd alimentation
    \item \verb-VssIn- : Creation of the vss alimentation
    \item \verb-Signal- : Creation of an internal net
\end{itemize}
        
\subsubsection{Parameters}

All kind of constructors have the same parameters :
\begin{itemize}
    \item \verb-name- : the name of the net (mandatory argument)
    \item \verb-arity- : the arity of the net (mandatory argument)
    \item \verb-indice- : for buses only : the LSB bit (optional argument : set to 0 by default)
\end{itemize}

\indent Only \verb-CkIn-, \verb-VddIn- and \verb-VssIn- do not have the same parameters : there is only the \verb-name- parameter (they are 1 bit nets).
    
\subsubsection{Example}

You can see a concrete example at : \hyperref[ref]{\emph{A concrete example}}{}{Example}{secexample}
    
\subsubsection{Errors}
    
Some errors may occur :
\begin{itemize}
    \item \verb-Error in LogicIn :-\\\verb-the lenght of the net must be a positive value.-\\One can not create a net with a negative lenght.
\end{itemize}

\subsubsection{See Also}

\hyperref[ref]{\emph{Introduction}}{}{Introduction}{secintroduction}
\hyperref[ref]{\emph{Alias}}{}{Alias}{secalias}
\hyperref[ref]{\emph{Extand}}{}{Extand}{secextend}
\hyperref[ref]{\emph{Cat}}{}{Cat}{seccat}
