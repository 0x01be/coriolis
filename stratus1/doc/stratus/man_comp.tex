\subsubsection{Name}

Eq/Ne : Easy way to test the value of the nets

\subsubsection{Synopsys}

\begin{verbatim}
netOut <= net.Eq ( "n" )
\end{verbatim}
  
\subsubsection{Description}

Comparaison functions are listed below :
\begin{itemize}
    \item \verb-Eq- : returns \verb-true- if the value of the net is equal to \verb-n-.
    \item \verb-Ne- : returns \verb-true- if the value of the net is different from \verb-n-.
\end{itemize}
\indent Note that it is possible to change the generator instanciated with the \verb-SetComp- method.

\subsubsection{Parameters}

The constant given as argument must be a string representing :
\begin{itemize}
    \item A decimal number
    \item A binary number : the string must begin with "0b"
    \item An hexadecimal number : the string must begin with "0x"
\end{itemize}    

\subsubsection{Example}

\begin{verbatim}
class essai ( Model ) :

  def Interface ( self ) :
    self.A = SignalIn  ( "a", 4 )
    
    self.S = SignalOut ( "s", 1 )
    self.T = SignalOut ( "t", 1 )

    self.vdd = VddIn  ( "vdd" )
    self.vss = VssIn  ( "vss" )
	
  def Netlist ( self ) :

    self.S <= self.A.Eq ( "4" )

    self.T <= self.A.Ne ( "1" )
\end{verbatim}
    
\subsubsection{Errors}
    
Some errors may occur :
\begin{itemize}
    \item \verb-[Stratus ERROR] Eq :-\\\verb-the number does not match with the net's lenght.-\\When one uses comparaison functions on one net, one has to check that the number corresponds to the size of the net.
    \item \verb-[Stratus ERROR] Eq :-\\\verb-the argument must be a string representing a number in decimal,-\\\verb-binary (0b) or hexa (0x).-\\The string given as argument does not have the right form.
\end{itemize}

\begin{htmlonly}

\subsubsection{See Also}

\hyperref[ref]{\emph{Introduction}}{}{Introduction}{secintroduction}
\hyperref[ref]{\emph{Netlist}}{}{Netlist}{secnetlist}
\hyperref[ref]{\emph{Instanciation of a multiplexor}}{}{Multipliexor}{secmux}
\hyperref[ref]{\emph{Instanciation of a shifter}}{Shifter}{}{secshift}
\hyperref[ref]{\emph{Instanciation of a register}}{}{Reg}{secreg}
\hyperref[ref]{\emph{Instanciation of constants}}{Constant}{}{secconstant}
\hyperref[ref]{\emph{Boolean operations}}{}{Boolean}{secbool}
\hyperref[ref]{\emph{Arithmetical operations}}{}{Arithmetic}{secarithmetic}

\end{htmlonly}
