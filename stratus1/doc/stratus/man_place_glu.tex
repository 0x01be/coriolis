\subsubsection{Name}

PlaceGlue -- Automatic placement of non placed instances

\subsubsection{Synopsys}

\begin{verbatim}
PlaceGlue ( cell )
\end{verbatim}

\subsubsection{Description}

This function places, thanks to the automatic placer Mistral of Coriolis, all the non placed instances of the cell.
    
\subsubsection{Parameters}

\begin{itemize}
    \item \verb-cell- : the cell which the fonction is applied to
\end{itemize}
    
%\subsubsection{Example}
%
%\begin{verbatim}
%PlaceGlue ( core )
%\end{verbatim}
%

\begin{htmlonly}

\subsubsection{See Also}

\hyperref[ref]{\emph{Introduction}}{}{Introduction}{secintroduction}
\hyperref[ref]{\emph{Layout}}{}{Layout}{seclayout}
\hyperref[ref]{\emph{PlaceCentric}}{}{PlaceCentric}{seccentric}
\hyperref[ref]{\emph{FillCell}}{}{FillCell}{secfillcell}
\hyperref[ref]{\emph{Pads}}{}{Pads}{secpads}
\hyperref[ref]{\emph{Alimentation rails}}{}{Alimentation rails}{secrails}
\hyperref[ref]{\emph{Alimentation connectors}}{}{Alimentation connectors}{secconnectors}
\hyperref[ref]{\emph{PowerRing}}{}{PowerRing}{secpowerring}
\hyperref[ref]{\emph{RouteCk}}{}{RouteCk}{secrouteck}

\end{htmlonly}
