\subsubsection{Name}

SignalIn, SignalOut ... -- Creation of nets

\subsubsection{Synopsys}

\begin{verbatim}
netA = SignalIn ( "a", 4 )
\end{verbatim}

\subsubsection{Description}

How to create and use nets.

\subsubsection{Nets}

Differents kind of nets are listed below :
\begin{itemize}
    \item \verb-SignalIn- : Creation of an input port
    \item \verb-SignalOut- : Creation of an output port
    \item \verb-SignalInOut- : Creation of an inout port
    \item \verb-SignalUnknown- : Creation of an input/output port which direction is not defined
    \item \verb-TriState- : Creation of a tristate port
    \item \verb-CkIn- : Creation of a clock port
    \item \verb-VddIn- : Creation of the vdd alimentation
    \item \verb-VssIn- : Creation of the vss alimentation
    \item \verb-Signal- : Creation of an internal net
\end{itemize}
        
\subsubsection{Parameters}

All kind of constructors have the same parameters :
\begin{itemize}
    \item \verb-name- : the name of the net (mandatory argument)
    \item \verb-arity- : the arity of the net (mandatory argument)
    \item \verb-indice- : for bit vectors only : the LSB bit (optional argument : set to 0 by default)
\end{itemize}

\indent Only \verb-CkIn-, \verb-VddIn- and \verb-VssIn- do not have the same parameters : there is only the \verb-name- parameter (they are 1 bit nets).
    
\subsubsection{Functions and methods}

Some functions/methods are provided in order to handle nets :
\begin{itemize}
    \item function \verb-Cat- : Concatenation of nets, beginning with the MSB
\begin{verbatim}
Inst ( 'DpgenInv'
     , map = { 'i0'  : Cat ( A, B )
             , 'nq'  : S
             , 'vdd' : vdd
             , 'vss' : vss
             }
     )
\end{verbatim}
\indent Or :
\begin{verbatim}
tab = []
tab.append ( A )
tab.append ( B )

Inst ( 'DpgenInv'
     , map = { 'i0'  : Cat ( tab )
             , 'nq'  : S
             , 'vdd' : vdd
             , 'vss' : vss
             }
     )
\end{verbatim}
\indent If A and B are 2 bits nets, the net \verb-myNet- will be such as :
\begin{verbatim}
myNet[3] = A[1]
myNet[2] = A[0]
myNet[1] = B[1]
myNet[0] = B[0]
\end{verbatim}
    \item function \verb-Extend- : Creation of a net which is an extension of the net which it is applied to
\begin{verbatim}
temp    = Signal (     "temp", 5 )
tempExt = Signal ( "temp_ext", 8 )

tempExt <= temp.Extand ( 8, 'one' )
\end{verbatim}
    \item method \verb-Alias- : Creation of an alias name for a net
\begin{verbatim}
cin.Alias  ( c_temp[0] )
cout.Alias ( c_temp[4] )
for i in range ( 4 ) :
  Inst ( "Fulladder"
       , map = { 'a'    : a[i]
               , 'b'    : b[i]
               , 'cin'  : c_temp[i]
               , 'sout' : sout[i]
               , 'cout' : c_temp[i+1]
               , 'vdd'  : vdd
               , 'vss'  : vss
               }
       )    
\end{verbatim}
\end{itemize}

\begin{htmlonly}

\subsubsection{Example}

You can see a concrete example at : \hyperref[ref]{\emph{Example}}{}{Example}{secexample}

\end{htmlonly}
    
\subsubsection{Errors}
    
Some errors may occur :
\begin{itemize}
    \item \verb-Error in SignalIn :-\\\verb-the lenght of the net must be a positive value.-\\One can not create a net with a negative lenght.
\end{itemize}

\begin{htmlonly}
\subsubsection{See Also}

\hyperref[ref]{\emph{Introduction}}{}{Introduction}{secintroduction}

\end{htmlonly}
