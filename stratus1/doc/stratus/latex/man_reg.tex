\subsubsection{Name}

Reg -- Easy way to instantiate a register

\subsubsection{Synopsys}

\begin{verbatim}
netOut <= netCk.Reg ( netIn )
\end{verbatim}
  
\subsubsection{Description}

This method is a method of net. The net which this method is applied to is the clock of the register. The net given as parameter is the input net. The method returns a net : the output net.\\
\indent Note that it is possible to change the generator instanciated with the \verb-SetReg- method.

\subsubsection{Example}

\begin{verbatim}
class essai ( Model ) :

  def Interface ( self ) :
    self.A  = SignalIn  (  "a", 4 )
    self.S  = SignalOut (  "s", 4 )

    self.Ck = CkIn ( "ck" )
    
    self.Vdd = VddIn  ( "vdd" )
    self.Vss = VssIn  ( "vss" )
	
  def Netlist ( self ) :

    self.S <= self.Ck.Reg ( self.A ) 
\end{verbatim}
    
\subsubsection{Errors}
    
Some errors may occur :
\begin{itemize}
    \item \verb-[Stratus ERROR] Reg : The input net does not have a positive arity.-\\The input net must have a positive arity.
    \item \verb-[Stratus ERROR] Reg : The clock does not have a positive arity.-\\The clock must have a positive arity.
\end{itemize}

\begin{htmlonly}

\subsubsection{See Also}

\hyperref[ref]{\emph{Introduction}}{}{Introduction}{secintroduction}
\hyperref[ref]{\emph{Netlist}}{}{Netlist}{secnetlist}
\hyperref[ref]{\emph{Instanciation of a multiplexor}}{}{Multiplexor}{secmux}
\hyperref[ref]{\emph{Instanciation of constants}}{}{Constant}{secconstant}
\hyperref[ref]{\emph{Boolean operations}}{}{Boolean}{secbool}
\hyperref[ref]{\emph{Arithmetical operations}}{}{Arithmetic}{secarithmetic}
\hyperref[ref]{\emph{Comparison operations}}{}{Comparison}{seccomp}

\end{htmlonly}
