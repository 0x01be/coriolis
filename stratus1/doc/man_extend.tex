\subsubsection{Name}

Extend -- Extention of nets

\subsubsection{Synopsys}

\begin{verbatim}
extendA = Signal ( "extend_a", 32 )

extendA <= netA.Extend ( 32, 'zero' ) 
\end{verbatim}

\subsubsection{Description}

This method creates a net which is an extension of the net which it is applied to. The number given as parameter has to be greater than the arity of the net.

\subsubsection{Parameters}

\begin{itemize}
    \item \verb-n- : the number which is the arity of the net
    \item \verb-type- : the type of the extension
    \begin{itemize}
        \item 'zero' : the extension nets are set to zero
        \item 'one' : the extension nets are set to one
        \item 'signed' : the extension nets are set thanks to the value of the msb bit
    \end{itemize}
\end{itemize}

\subsubsection{Example}

\begin{verbatim}
temp    = Signal (     "temp", 5 )
tempExt = Signal ( "temp_ext", 8 )

tempExt <= temp.Extand ( 8, 'one' )
\end{verbatim}

\indent The arity of the net \verb-tempExt- is 8 and 3 MSB of the net are "1".
  
\subsubsection{Errors}
    
Some errors may occur :
\begin{itemize}
    \item \verb-[Stratus ERROR] Extend :-\\\verb-the net can not be extended to n bits, it's arity is already m.-\\The number one wants must be greater than the arity of the net. Otherwise, the net can not be extended.
\end{itemize}
        
\subsubsection{See Also}

\hyperref[ref]{\emph{Introduction}}{}{Introduction}{secintroduction}
\hyperref[ref]{\emph{Nets}}{}{Nets}{secnet}
\hyperref[ref]{\emph{Alias}}{}{Alias}{secalias}
\hyperref[ref]{\emph{Cat}}{}{Cat}{seccat}
