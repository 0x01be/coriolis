\subsubsection{Name}

PlaceTop -- Places an instance at the top of the "reference instance"

\subsubsection{Synopsys}

\begin{verbatim}
PlaceTop ( ins, sym, offset )
\end{verbatim}

\subsubsection{Description}

Placement of an instance.\\
\indent The instance has to be instantiated in the method \verb-Netlist- in order to use the \verb-PlaceTop- function.\\
    
\indent The bottom left corner of the abutment box of the instance is placed, after beeing symetrized and/or rotated, toward the top left corner of the abutment box of the "reference instance". The newly placed instance becomes the "reference instance".

\subsubsection{Parameters}

\begin{itemize}
    \item \verb-ins- : Instance to place.
    \item \verb-sym- : Geometrical operation to be performed on the instance before beeing placed.\\The \verb-sym- argument can take eight legal values :
    \begin{itemize}
        \item \verb-NOSYM- : no geometrical operation is performed
        \item \verb-SYM_Y- : Y becomes -Y, that means toward X axe symetry
        \item \verb-SYM_X- : X becomes -X, that means toward Y axe symetry
        \item \verb-SYMXY- : X becomes -X, Y becomes -Y
        \item \verb-ROT_P- : a positive 90 degrees rotation takes place
        \item \verb-ROT_M- : a negative 90 degrees rotation takes place
        \item \verb-SY_RP- : Y becomes -Y, and then a positive 90 degrees rotation takes place
        \item \verb-SY_RM- : Y becomes -Y, and then a negative 90 degrees rotation takes place
    \end{itemize}
    \item \verb-offset- (optionnal) : An offset is put between this instance and the reference instance. The value given as argument must be a multiple of PITCH and represents the lenght of the hole between the instances
\end{itemize}

\subsubsection{Example}

\begin{verbatim}
Place    ( myInst1, NOSYM, 0, 0 )
PlaceTop ( myInst2, SYM_Y       )
\end{verbatim}

\subsubsection{Errors}
    
Some errors may occur :    
\begin{itemize}
    \item \verb-[Stratus ERROR] Placement : the instance doesn't exist.-\\The instance must be instanciated in order to be placed.
    \item \verb-[Stratus ERROR] Placement : the first argument is not an instance.-
    \item \verb-[Stratus ERROR] Placement : the instance is already placed.-\\One can not place an instance twice    
    \item \verb-[Stratus ERROR] PlaceTop : no previous instance.-\\One can use \verb-PlaceTop- only if a reference instance exist. Use a \verb-Place- call before. 
    \item \verb-[Stratus ERROR] PlaceTop : wrong argument for placement type.-\\The symetry given as argument is not correct.
\end{itemize}

\subsubsection{See Also}

\hyperref[ref]{\emph{Introduction}}{}{Introduction}{secintroduction}
\hyperref[ref]{\emph{Place}}{}{Place}{secplace}
\hyperref[ref]{\emph{PlaceBottom}}{}{PlaceBottom}{secbottom}
\hyperref[ref]{\emph{PlaceRight}}{}{PlaceRight}{secright}
\hyperref[ref]{\emph{PlaceLeft}}{}{PlaceLeft}{secleft}
\hyperref[ref]{\emph{SetRefIns}}{}{SetRefIns}{secsetrefins}
\hyperref[ref]{\emph{DefAb}}{}{DefAb}{secdefab}
\hyperref[ref]{\emph{ResizeAb}}{}{ResizeAb}{secresizeab}
